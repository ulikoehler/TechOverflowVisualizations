% tikzpic.tex
\documentclass[tikz, border=1mm]{standalone}
\usepackage[europeanresistors,americaninductors,americancurrents,siunitx]{circuitikz}
\usepackage{siunitx}
\begin{document}

%%% Coordinate definitions: %%%
% Position of the bridge (top upper corner)
\def\x{6}
\def\y{6}
% Size of the bridge
\def\dx{3}
\def\dy{3}

\begin{circuitikz}[american voltages]
  % Voltage source
  \draw (0,0) to [V, l_=5<\volt>] (0, \y) to (\x, \y);
  % Left half bridge
  \draw[dashed] (\x, \y) to [R, l=R1, *-*] (\x-\dx,\y-\dy); % Top left resistor
  \draw (\x-\dx,\y-\dy) to [R, l=R3, -*] (\x,\y-2*\dy);  % Bottom left resistor
  % Right half bridge
  \draw (\x,\y) to [R, l=R2, -*] (\x+\dx, \y-\dy); % Top right resistor
  \draw[dashed] (\x+\dx, \y-\dy) to [R, l=R4, -*] (\x,\y-2*\dy);  % Bottom right resistor
  % Draw connection to (-) terminal of voltage source
  \draw (\x,\y-2*\dy) to (\x, 0) to (0,0);

  % Draw voltmeter
  \draw (\x-\dx, \y-\dy) to [voltmeter] (\x+\dx, \y-\dy);

  % Draw burnout current sources
  \draw (\x-\dx, \y) to [I, i_=I1, -*] (\x-\dx, \y-\dy); % IN(+)
  \draw (\x+\dx, \y-\dy) to [I, i=I2] (\x+\dx, \y-2*\dy)
  to (\x,\y-2*\dy); % IN(-)

\end{circuitikz}

\end{document}